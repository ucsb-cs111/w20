\documentclass[11pt]{article}

\topmargin 0in
\textheight 9in
\oddsidemargin 0pt
\evensidemargin 0pt
\textwidth 6.5in

\newcommand{\numpy}{{\tt numpy}}    % tt font for numpy

\begin{document}

$$\mbox{\Large \bf CS 111: Review Quiz: Answer key} $$

\medskip


\begin{enumerate}

\item
$$ 
A^T = \left(
\begin{array}{ccc}
3 & 0 & 1 \\ 	
-1 & 1 & 0 \\ 
2 & 2 & -1 \\
\end{array} 
\right),
A^2 = \left(
\begin{array}{ccc}
11 & -4 & 2 \\
2 & 1 & 0 \\ 
2 & -1 & 3 \\
\end{array} 
\right),
A^TA = \left(
\begin{array}{ccc}
10 & -3 & 5 \\
-3 & 2 & 0 \\
5 & 0 & 9 \\
\end{array} 
\right).
$$

\item
$||(3,1,4,1,5)^T||_2 = \sqrt{52} \approx 7.2111$

\item
$$
\left(
\begin{array}{ccc}
2 & -3 & 1 \\
0 & 2 & 3 \\ 
1 & 0 & 1 \\
\end{array} 
\right) 
\times
\left(
\begin{array}{c}
 x_1 \\
 x_2 \\ 
 x_3 \\
\end{array} 
\right) 
=
\left(
\begin{array}{c}
 1 \\
 7 \\ 
 4 \\
\end{array} 
\right) 
$$
In \numpy, {\tt A = np.array([[2, -3, 1], [0, 2, 3], [1, 0, 1]])}
and {\tt b = np.array([1, 7, 4])}.
The 1-dimensional {\tt numpy} array $b$ can represent either a
column vector or a row vector; Python's matrix-vector multiplication
operator {\tt @} will do the right thing.

\item
$$x = 
\left(
\begin{array}{c}
 3 \\
 2 \\ 
 1 \\
\end{array} 
\right) 
$$

\item
There are many answers to this.  Here's one:
Let {\tt A = np.array([[1, 2], [2, 4]])} 
and {\tt b = np.array([3, 3])}.
(In math notation we write $b=(3,3)^T$, 
which is a column vector because of the transpose.)
Explanation 1 (column view): 
Matrix $A$ is singular, so the space spanned
by its columns is only one-dimensional, and it consists of
multiples of the vector $(1,2)^T$, which do not include $b$.
Explanation 2 (row view):
The two lines described by the rows of $Ax=b$ are parallel
and hence do not intersect.
Explanation 3 (brute force view):
No matter what $x$ is, the second entry of $Ax$ will be equal
to twice the first entry of $Ax$, which rules out $b$.

\item
There are many answers to this.  Here's one:
Take $A$ to be the same matrix as in the previous problem,
and $b = (3,6)^T$.
Two solutions are $x = (1,1)^T$ and $x = (3,0)^T$.

\item
No, it's not possible to have exactly two solutions to $Ax=b$.
If $x$ and $y$ are two different solutions, then there are
infinitely many solutions: $x + \alpha(y-x)$ is a solution
for every $\alpha$.

\item
$A$ has two eigenvalues, $3$ and $5$.
Any multiple of $(1,1)^T$ is an eigenvector corresponding to $3$,
and any multiple of $(1,-1)^T$ is an eigenvector corresponding to $5$.

\item
$f'(x)=21x^2-4x+4$.

\item
$\partial z / \partial x = e^{y/2}$, and
$\partial z / \partial y = (x/2)e^{y/2}$.

\item
$f(x) = x^3/3 - \cos x + c$ for some constant $c$ (any constant will do).

\item
The height is maximum when the derivative $dh/dt$ is zero.
$dh/dt = 1280-32t$, which is zero when $t = 40$, 
at which time the height is $h = 25600$ feet.
The bullet hits the ground when $h=0$,
which means $1280t = 16t^2$, which means $t=1280/16 = 80$ seconds after
firing.
(The other solution to $h=0$ is of course $t=0$.)

\item
$y=e^{x^2/2}$.  

\end{enumerate}

\end{document}
