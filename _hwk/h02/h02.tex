\documentclass[11pt]{article}

\newcommand{\numpy}{{\tt numpy}}    % tt font for numpy

\topmargin -.75in
\textheight 9in
\oddsidemargin -.25in
\evensidemargin -.25in
\textwidth 7in

\begin{document}

$$\mbox{\Large \bf CS 111: Homework 2: Due by 11:59pm Monday, January 20}$$

\medskip\noindent
{\bf Submit your homework online as a PDF file to GradeScope,
and tell GradeScope which page(s) contain each problem.}

\par\bigskip
{\bf 1.}
Consider the code for the temperature problem in 
the lecture file {\tt temperature.py}, 
especially the routines {\tt make\_A()} and {\tt make\_b()}
that create the matrix $A$ and right-hand side $b$.
Experiment with different ways of setting the boundary conditions,
which are the parameters {\tt top}, {\tt bottom}, {\tt left}, and {\tt right} 
to {\tt make\_b()}.
Make a plot of the most interesting result that you get (in your opinion), 
and explain how you got it. 
If you want, you can also experiment with {\tt matplotlib} 
to make a more interesting plot of your result. 
(The CS 111 logo on the course web page was obtained this way in 2010; 
maybe we can get a new logo this year!)

\par\bigskip
{\bf 2.}
Again consider the routines {\tt make\_A()} and {\tt make\_b()}
that create the matrix~$A$ and right-hand side~$b$ for the temperature problem.
Let $k=100$.

\par\medskip
{\bf 2a.}
How many elements are there in $b$?

\par\medskip
{\bf 2b.}
Considering all possible choices for the temperatures on the boundary,
what is the largest number of elements of $b$ that could possibly 
be nonzero? 

\par\medskip
{\bf 2c.}
Explain why the rest of the elements of $b$ are zero, no matter
what the boundary temperatures are.

\par\bigskip
{\bf 3.}
Write the following matrix in the form $A=LU$, 
where $L$ is a unit lower triangular matrix
(that is, a lower triangular matrix with ones on the diagonal) 
and $U$ is an upper triangular matrix.
$$A =
   \left(
   \begin{array}{ccc}
    4 & -1 & -1 \\ 	
   -1 &  4 & -1 \\ 
   -1 & -1 &  4 \\
   \end{array} \right)
$$

\par\bigskip
{\bf 4.}
The following three statements are all {\bf false}. For each one, 
give a counterexample consisting of a 3-by-3 matrix or matrices, 
and show the computation that proves that the statement fails.

\par\medskip
{\bf 4a.}
If $P$ is a permutation matrix and $A$ is any matrix, then $PA=AP$.

\par\medskip
{\bf 4b.}
If matrix $A$ is nonsingular, then it has a factorization $A=LU$
where $L$ is lower triangular and $U$ is upper triangular.

\par\medskip
{\bf 4c.}
The product of two symmetric matrices is a symmetric matrix.

\newpage
\par\bigskip
{\bf 5a.} Consider the permutation matrix 
$$P =
   \left(
   \begin{array}{cccc}
    0 & 1 & 0 & 0 \\ 	
    0 & 0 & 0 & 1 \\ 	
    0 & 0 & 1 & 0 \\ 	
    1 & 0 & 0 & 0 \\ 	
   \end{array} \right)
$$
Find a 4-element permutation vector {\tt v = np.array(something)}
such that, for {\em every}\, 4-by-4 matrix $A$, 
we have {\tt A[v,:] == P @ A}.
Test your answer by running a few lines of Python, 
and turn in the result.

\par\medskip
{\bf 5b.} For the same $P$, 
find a 4-element permutation vector {\tt w = np.array(something)}
such that, for {\em every}\, 4-by-4 matrix $A$, 
we have {\tt A[:,w] == A @ P}. 
Test your answer and turn in the result.

\par\bigskip
{\bf 6.}
Write {\tt Usolve()}, analogous to {\tt Lsolve()} in 
the lecture file {\tt LU.py},
to solve an upper triangular system $Ux=y$. 
Warning: Notice that, unlike in {\tt Lsolve()}, 
the diagonal elements of $U$ don't have to be equal to one.
Test your answer, both by itself and with {\tt LUsolve()},
and turn in the result.
Hint: Loops can be run backward in Python, 
say from $n-1$ down to $0$, by writing
$$\mbox{\tt for i in reversed(range(n)):}$$

\end{document}
