\documentclass[11pt]{article} 

\usepackage{amssymb,amsmath,tikz}

\newcommand{\numpy}{{\tt numpy}}            % tt font for numpy
\newcommand{\scipy}{{\tt scipy}}            % tt font for scipy
\newcommand{\matplotlib}{{\tt matplotlib}}  % tt font for matplotlib

% \topmargin -1in
% \textheight 9in
% \oddsidemargin  -.25in
% \evensidemargin -.20in
% \textwidth 7in
\topmargin -.5in
\textheight 7.5in
\oddsidemargin  -0.25in
\evensidemargin -0.20in
\textwidth 7in

\begin{document}

$$\mbox{\Large \bf CS 111: Homework 8: Due by 11:59 pm Wednesday, March 11, 2020}$$
\par\smallskip\noindent
{\bf Submit your paper as one PDF file,
and tell GradeScope which page(s) each problem is on.}

\par\bigskip
{\bf 1.}
This problem is about another connection between graphs, matrices,
and eigenvalues.
This time the graph in question is undirected, with no arrows on its edges.
Let $G$ be an undirected graph with $n$ vertices, 
which we take to be the integers 0 through $n-1$.
An edge of $G$ is an unordered pair of integers $(i,j)$.
We assume that $G$ has no multiple edges 
(that is, edge $(i,j)$ only occurs once)
and no loops (that is, no edges $(i,i)$).

The {\em Laplacian matrix} of $G$ is the $n$-by-$n$ matrix $L$ 
whose diagonal element $L[i, i]$ is the number of neighbors of vertex $i$
(also called the {\em degree} of vertex $i$),
and whose off-diagonal element $L[i, j]$ 
is $-1$ if $(i,j)$ is an edge of $G$ and is $0$ otherwise.
For example, the Laplacian matrix of the graph consisting of 4 vertices
connected in a square (also called a 4-cycle) is
\par\smallskip\noindent

\begin{minipage}{.45\textwidth}
$$ 
L = \left(
\begin{array}{cccc}
 2 & -1 &  0 & -1 \\    
-1 &  2 & -1 &  0 \\    
 0 & -1 &  2 & -1 \\    
-1 &  0 & -1 &  2 \\    
\end{array} 
\right),
$$
\end{minipage}
\hspace{0.4in}
where 
\hspace{0.03in}
$G=$
\begin{minipage}{.45\textwidth}
\begin{tikzpicture}
  [scale=.8,auto=left,every node/.style={circle}]
  \node (n0) at (2,4) {0};
  \node (n1) at (4,4) {1};
  \node (n2) at (4,2) {2};
  \node (n3) at (2,2) {3};
  \foreach \from/\to in {n0/n1,n1/n2,n2/n3,n3/n0}
    \draw (\from) -- (\to);
\end{tikzpicture}
\end{minipage}

\par\smallskip\noindent
The Laplacian matrix is symmetric because the graph is undirected.
It's a theorem that all the eigenvalues of any symmetric real matrix are real numbers,
and it's also a theorem that all the eigenvalues of the Laplacian matrix
of a graph are greater than or equal to zero.

\par\medskip
{\bf 1.1.}
Use {\tt linalg.eigh()} (note the h!) to find the four eigenvalues of the 
Laplacian matrix of the 4-cycle as above.
You should find that they are all nonnegative real numbers,
and that one of them is equal to zero.

\par\medskip
{\bf 1.2.}
In fact, zero is an eigenvalue of the Laplacian matrix of every graph.
Prove this by exhibiting an $n$-vector $v^{(0)}$ that is an eigenvector 
for the eigenvalue zero for {\em every} $n$-vertex graph, 
and explaining why $Lv^{(0)} = 0v^{(0)}$.

\par\medskip
{\bf 1.3.} 
Given an $n$-vertex graph $G$ and any $n$-vector $v$, 
we can think of the $n$~elements of the vector~$v$ as labels for the 
$n$~vertices of the graph; 
$v[0]$ labels vertex~0, $v[1]$ labels vertex~1, and so forth.
Used as labels in this way,
the eigenvectors of the Laplacian matrix of a graph are in some ways
analogous to the fundamental modes of vibration of a physical object.

For example, let $P_n$ be the graph with $n$ vertices joined in a single path, 
so that $P_n$ has the $n-1$ edges $\{(0,1), (1,2), \ldots, (n-2, n-1)\}$.
Write a python function {\tt path(n)} that computes the $n$-by-$n$ Laplacian matrix
$L_n$ of the graph $P_n$ as a \numpy\ array.
Show your function, and show its output for $n=5$ as an example.
Also show the output of {\tt linalg.eigh()} on $L_5$.

Now we'll see how the Laplacian eigenvectors of the graph $P_n$ correspond 
to ``modes of vibration.'' 
The idea is to think of the path $P_n$ as a violin string.
Use your function {\tt path()} to compute the Laplacian matrix $L_{100}$ of 
the graph $P_{100}$. 
Then use {\tt linalg.eigh()} to compute the eigenvalues $d_0,\ldots,d_{99}$
and eigenvectors $v^{(0)},\ldots,v^{(99)}$ of the matrix $L_{100}$.
Check to see whether the eigenvalues come out of {\tt linalg.eigh()} in
increasing order of size; 
if not, reorder both the eigenvalues and eigenvectors so that
$d_0 \le d_1 \le \cdots \le d_{99}$.
Don't print out $L_{100}$ or the lists of eigenvalues and eigenvectors, 
but make and turn in the following three plots (all nicely labeled, of course):
\begin{itemize}
\item Plot the 100 elements $v^{(0)}_0, v^{(0)}_1, \ldots, v^{(0)}_{99}$
      of the eigenvector $v^{(0)}$ (this is a pretty simple picture).
\item Make one plot that has 4 lines on it, plotting the elements of 
      each eigenvector $v^{(1)}$ through $v^{(4)}$.
\item Plot the 100 eigenvalues $d_i$ versus $i$.
\end{itemize}

\par\medskip
{\bf 1.4.} 
How close is the temperature matrix to being a Laplacian matrix?
Specifically, which (and how many) of the nonzero elements of the 
2D temperature matrix with $k=100$ would you need to change to make 
it into the Laplacian matrix of some graph? 
Can you describe in words what that graph is?

\par\bigskip
{\bf 2.} 
An important problem in classical mechanics is to determine the motion
of two bodies under mutual gravitational attraction.
Suppose that a body of mass $m$ is orbiting a second body of much larger mass $M$,
such as the earth orbiting the sun.
From Newton's laws of motion and gravitation,
the orbital trajectory $(x_0(t), x_1(t))$ is described by the system
of second-order ODEs
\begin{align}
\ddot x_0 &= -GMx_0/r^3, \\
\ddot x_1 &= -GMx_1/r^3, 
\end{align}
where $G$ is the gravitational constant and $r = (x_0^2 + x_1^2)^{1/2} = ||x||$
is the distance of the orbiting body from the center of mass of the two bodies.
For this experiment, we choose units such that $GM=1$.

\par\medskip
{\bf 2.1.}
Use {\tt integrate.solve\_ivp()} to solve this system of ODEs with the initial conditions
\begin{align}
x_0(0)      = 1-\epsilon, & \;\;\; x_1(0)      = 0, \\
\dot x_0(0) = 0,   & \;\;\; \dot x_1(0) = \Big(\frac{1+\epsilon}{1-\epsilon}\Big)^{1/2},
\end{align}
where $\epsilon$ is the eccentricity of the resulting elliptical orbit,
which has period $2\pi$.
Try the values $\epsilon=0$ (which should give a circular orbit),
$\epsilon=0.5$, and $\epsilon=0.9$.
For each case, solve the ODE for at least one orbital period and obtain output 
at enough intermediate points to draw a smooth plot of the orbital trajectory.
Make separate plots of $x_0$ versus $t$, $x_1$ versus $t$, and $x_0$ versus $x_1$,
all with well-labeled axes and clear titles.
For your plot of the orbit itself, $x_0$ versus  $x_1$, 
use {\tt plt.gca().axis('equal')} to make sure the scale is the same on both axes,
so that a circle will look like a circle.

Experiment with different error tolerances 
(use {\tt help(integrate.solve\_ivp)} to find out how to set error tolerances)
to see how they affect 
(i)  the amount of time required for the solution and 
(ii) how close the orbit comes to being closed.
If you trace the orbit through several periods, 
does the orbit tend to wander or remain steady?
In addition to your plots, 
turn in an explanation in English of what experiments you did,
what you observed, and what your conclusions were.

\par\medskip
{\bf 2.2.}
Physics tells us that the orbit described by this ODE
(like any closed physical system) should conserve both energy and angular momentum.
Check your numerical solutions from part (a) to see how well the following
quantities remain constant throughout time.

\par\medskip
Conservation of energy:
$$\frac{\dot x_0^2 + \dot x_1^2}{2} - \frac{1}{r}$$

Conservation of angular momentum:
$$x_0\dot x_1 - x_1\dot x_0$$

Present your results using whatever numbers or graphs you think
are most informative, and describe your conclusions in English.

\end{document}
