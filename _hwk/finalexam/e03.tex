\documentclass[11pt]{article} 

\usepackage{amssymb,amsmath}

\newcommand{\numpy}{{\tt numpy}}            % tt font for numpy
\newcommand{\scipy}{{\tt scipy}}            % tt font for scipy
\newcommand{\matplotlib}{{\tt matplotlib}}  % tt font for matplotlib

\topmargin -1in
\textheight 9in
\oddsidemargin  -.25in
\evensidemargin -.20in
\textwidth 7in

\begin{document}

$$\mbox{\Large \bf CS 111: Optional Exam: Due by 11:59pm Monday, April 13}$$
\par\bigskip\noindent

This optional exam replaces the final exam for CS 111 in winter quarter 2020.
You may take the exam or not as you please. 
If you take the exam, 
it will not lower your currently recorded final grade for CS 111;
see Piazza {\tt @244} for details.
The exam is open-book, but you may not consult or discuss it with anyone else.
You must submit your work online as a PDF file to GradeScope.

The exam consists of one somewhat open-ended question that uses what 
we've studied in CS 111 in a standard model of epidemics.
You will first do a straightforward {\tt scipy} simulation with a plot
and an explanation of your work, 
and then you are asked to explore beyond the straightforward simulation.
The correct answer to the straightforward part will be graded as a B (80-89).
An A grade (90-100) will be given for further exploration that produces
interesting results and well-explained insight.

\par\bigskip
\noindent
{\bf Question 0.}
Do you agree to the following?
\begin{quotation}
\noindent
I will not communicate anything about this exam with anyone 
other than the instructor before the exam is due.
I understand that failure to abide by this declaration constitutes 
academic dishonesty under the UCSB Student Conduct Code.
\end{quotation}
\noindent
If so, please answer ``I agree'' on your exam paper.
Otherwise, please do not continue with the exam.

\par\bigskip
\noindent
{\bf Question 1.} 
Medical researchers have used ODEs as a tool to model the course of epidemics.
If you're interested (not required for the exam), 
you can learn much more about ODE epidemic models in this survey paper:
{\tt https://epubs.siam.org/doi/abs/10.1137/S0036144500371907}.
Here we will consider a somewhat simplified version of a 
widely used model due to Kermack and McKendrick.
The model is
\begin{align}
\dot y_0 &= -cy_0y_1, \\
\dot y_1 &= cy_0y_1 - ry_1 - fy_1, \\
\dot y_2 &= ry_1, \\
\dot y_3 &= fy_1,
\end{align}
where $y_0$ is the number of susceptible individuals,
$y_1$ represents infective individuals,
$y_2$ represents immune individuals and infectives 
removed by recovery and immunity,
and $y_3$ represents infectives removed by fatality.
The parameters $c$, $r$, and $f$ represent the contaigon rate,
the recovery rate, and the fatality rate, all per unit time.
(The time and the rates are expressed in units chosen to make the 
equations simple, not to model any specific real situation.)

Use {\tt integrate.solve\_ivp()} to solve this system numerically,
with the parameter values $c=1$, $r=5$, and $f=1$,
and initial values $y_0(0)=95$, $y_1(0)=5$, and $y_2(0)=y_3(0)=0$.
Solve for times $t=0$ to $t=1$.
Plot all four components of $y$ on the same graph as a function of $t$,
using {\tt plt.legend()} to show which curve is which,
and labeling the plot axes appropriately.
As expected with an epidemic, 
you should see the number of infections grow at first,
then diminish to zero.

Here is the open-ended part of the problem.
Experiment with other values for the rate parameters ($c$, $r$, and $f$),
and with other initial conditions (population size, number of initially
infected individuals, number of initially immune individuals),
using different time intervals if necessary to see how things stabilize.
Which parameters have larger or smaller effects on the course of the epidemic?
Can you find values for which the epidemic never infects the whole population?
What other interesting values can you find or describe?
What other observations can you make?

Turn in all your python/scipy code (from Jupyter), plots of your results, 
and explanations in English of what you tried and what you discovered.

\end{document}
