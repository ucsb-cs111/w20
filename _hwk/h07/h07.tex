\documentclass[11pt]{article} 

\usepackage{amssymb,amsmath}

\newcommand{\numpy}{{\tt numpy}}            % tt font for numpy
\newcommand{\scipy}{{\tt scipy}}            % tt font for scipy
\newcommand{\matplotlib}{{\tt matplotlib}}  % tt font for matplotlib

% \topmargin -1in
% \textheight 9in
% \oddsidemargin  -.25in
% \evensidemargin -.20in
% \textwidth 7in
\topmargin -.5in
\textheight 7.5in
\oddsidemargin  -0.25in
\evensidemargin -0.20in
\textwidth 7in

\begin{document}

$$\mbox{\Large \bf CS 111: Homework 7: Due by 11:59 pm Wednesday, March 4, 2020}$$
\par\smallskip\noindent
{\bf Submit your paper as one PDF file,
and tell GradeScope which page(s) each problem is on.}

\par\bigskip
{\bf 1.}
The standard form of a first-order ODE initial value problem is
$$ \dot y = f(t, y), \;\; y(t0) = y0, $$
where $t$ is a scalar and $y$ is a vector.
Write each of the following ODEs as an equivalent first-order system
of ODEs in standard form:

\par\bigskip
{\bf 1.1.} Van der Pol equation:
$$ \frac{d^2 x}{dt^2} = (1-x^2)\frac{dx}{dt} - x . $$

\par\bigskip
{\bf 1.2.} Blasius equation:
$$ \frac{d^3 x}{dt^3} = -x\frac{dx}{dt}. $$

\par\bigskip
{\bf 1.3.} Newton's second law of motion for a two-body problem in 2 dimensions 
($G$ and $M$ are constants):
\begin{align}
\frac{d^2 x_0}{dt^2} &= -GM\frac{x_0}{(x_0^2 + x_1^2)^{3/2}}, \\
\frac{d^2 x_1}{dt^2} &= -GM\frac{x_1}{(x_0^2 + x_1^2)^{3/2}}.
\end{align}

\par\bigskip
{\bf 2.} (Compare NCM problem 7.16.)
This problem is partly about ODEs and partly about making nice plots with
\matplotlib\ (we import {\tt matplotlib.pyplot} as {\tt plt}).

Many modifications of the Lotka--Volterra predator-prey model
that we saw in class on February 25 have been proposed to
more accurately reflect what happens in nature.
For example, the number of rabbits can be prevented from growing
indefinitely by fixing a maximum number $R$ and changing the equations to
\begin{align}
\frac{dr}{dt} &= 2\Big(1-\frac{r}{R}\Big)r - \alpha rf, \\
\frac{df}{dt} &= -f + \alpha rf,
\end{align}
where $t$ is time, $r(t)$ is the number of rabbits, 
$f(t)$ is the number of foxes, and $\alpha>0$ is a constant.
This makes $dr/dt$ negative whenever $r > R$, 
which guarantees that the number of rabbits can never grow to exceed $R$.

For $\alpha = 0.01$, compare the behavior of the original model
with the behavior of this modified model with $R = 400$.
Solve the equations (using {\tt integrate.solve\_ivp()} as we did in class)
over 50 units of time, 
assuming that there are initially 300 rabbits and 150 foxes.
Make four different plots to show the solutions and the phase
space diagrams for both models as follows:
\begin{itemize}
\item number of foxes and rabbits (on the same plot) versus time for the original model,
\item number of foxes and rabbits (on the same plot) versus time for the modified model,
\item number of foxes versus number of rabbits (phase space) for the original model,
\item number of foxes versus number of rabbits (phase space) for the modified model.
\end{itemize}
For all plots, label all curves (with {\tt plt.legend()}) and all axes,
and put a title on each plot that identifies it clearly.
For the phase space plots, set the aspect
ratio so that equal increments on the $x$- and $y$-axes are equal in size.
(You may find the \matplotlib\ tutorial linked under the ``help'' menu
in Jupyter useful.)

\end{document}
